
%% bare_jrnl.tex
%% V1.4b
%% 2015/08/26
%% by Michael Shell
%% see http://www.michaelshell.org/
%% for current contact information.
%%
%% This is a skeleton file demonstrating the use of IEEEtran.cls
%% (requires IEEEtran.cls version 1.8b or later) with an IEEE
%% journal paper.
%%
%% Support sites:
%% http://www.michaelshell.org/tex/ieeetran/
%% http://www.ctan.org/pkg/ieeetran
%% and
%% http://www.ieee.org/

%%*************************************************************************
%% Legal Notice:
%% This code is offered as-is without any warranty either expressed or
%% implied; without even the implied warranty of MERCHANTABILITY or
%% FITNESS FOR A PARTICULAR PURPOSE! 
%% User assumes all risk.
%% In no event shall the IEEE or any contributor to this code be liable for
%% any damages or losses, including, but not limited to, incidental,
%% consequential, or any other damages, resulting from the use or misuse
%% of any information contained here.
%%
%% All comments are the opinions of their respective authors and are not
%% necessarily endorsed by the IEEE.
%%
%% This work is distributed under the LaTeX Project Public License (LPPL)
%% ( http://www.latex-project.org/ ) version 1.3, and may be freely used,
%% distributed and modified. A copy of the LPPL, version 1.3, is included
%% in the base LaTeX documentation of all distributions of LaTeX released
%% 2003/12/01 or later.
%% Retain all contribution notices and credits.
%% ** Modified files should be clearly indicated as such, including  **
%% ** renaming them and changing author support contact information. **
%%*************************************************************************


% *** Authors should verify (and, if needed, correct) their LaTeX system  ***
% *** with the testflow diagnostic prior to trusting their LaTeX platform ***
% *** with production work. The IEEE's font choices and paper sizes can   ***
% *** trigger bugs that do not appear when using other class files.       ***                          ***
% The testflow support page is at:
% http://www.michaelshell.org/tex/testflow/



\documentclass[journal]{IEEEtran}
%
% If IEEEtran.cls has not been installed into the LaTeX system files,
% manually specify the path to it like:
% \documentclass[journal]{../sty/IEEEtran}





% Some very useful LaTeX packages include:
% (uncomment the ones you want to load)


% *** MISC UTILITY PACKAGES ***
%
%\usepackage{ifpdf}
% Heiko Oberdiek's ifpdf.sty is very useful if you need conditional
% compilation based on whether the output is pdf or dvi.
% usage:
% \ifpdf
%   % pdf code
% \else
%   % dvi code
% \fi
% The latest version of ifpdf.sty can be obtained from:
% http://www.ctan.org/pkg/ifpdf
% Also, note that IEEEtran.cls V1.7 and later provides a builtin
% \ifCLASSINFOpdf conditional that works the same way.
% When switching from latex to pdflatex and vice-versa, the compiler may
% have to be run twice to clear warning/error messages.






% *** CITATION PACKAGES ***
%
%\usepackage{cite}
% cite.sty was written by Donald Arseneau
% V1.6 and later of IEEEtran pre-defines the format of the cite.sty package
% \cite{} output to follow that of the IEEE. Loading the cite package will
% result in citation numbers being automatically sorted and properly
% "compressed/ranged". e.g., [1], [9], [2], [7], [5], [6] without using
% cite.sty will become [1], [2], [5]--[7], [9] using cite.sty. cite.sty's
% \cite will automatically add leading space, if needed. Use cite.sty's
% noadjust option (cite.sty V3.8 and later) if you want to turn this off
% such as if a citation ever needs to be enclosed in parenthesis.
% cite.sty is already installed on most LaTeX systems. Be sure and use
% version 5.0 (2009-03-20) and later if using hyperref.sty.
% The latest version can be obtained at:
% http://www.ctan.org/pkg/cite
% The documentation is contained in the cite.sty file itself.






% *** GRAPHICS RELATED PACKAGES ***
%
\ifCLASSINFOpdf
  % \usepackage[pdftex]{graphicx}
  % declare the path(s) where your graphic files are
  % \graphicspath{{../pdf/}{../jpeg/}}
  % and their extensions so you won't have to specify these with
  % every instance of \includegraphics
  % \DeclareGraphicsExtensions{.pdf,.jpeg,.png}
\else
  % or other class option (dvipsone, dvipdf, if not using dvips). graphicx
  % will default to the driver specified in the system graphics.cfg if no
  % driver is specified.
  % \usepackage[dvips]{graphicx}
  % declare the path(s) where your graphic files are
  % \graphicspath{{../eps/}}
  % and their extensions so you won't have to specify these with
  % every instance of \includegraphics
  % \DeclareGraphicsExtensions{.eps}
\fi
% graphicx was written by David Carlisle and Sebastian Rahtz. It is
% required if you want graphics, photos, etc. graphicx.sty is already
% installed on most LaTeX systems. The latest version and documentation
% can be obtained at: 
% http://www.ctan.org/pkg/graphicx
% Another good source of documentation is "Using Imported Graphics in
% LaTeX2e" by Keith Reckdahl which can be found at:
% http://www.ctan.org/pkg/epslatex
%
% latex, and pdflatex in dvi mode, support graphics in encapsulated
% postscript (.eps) format. pdflatex in pdf mode supports graphics
% in .pdf, .jpeg, .png and .mps (metapost) formats. Users should ensure
% that all non-photo figures use a vector format (.eps, .pdf, .mps) and
% not a bitmapped formats (.jpeg, .png). The IEEE frowns on bitmapped formats
% which can result in "jaggedy"/blurry rendering of lines and letters as
% well as large increases in file sizes.
%
% You can find documentation about the pdfTeX application at:
% http://www.tug.org/applications/pdftex





% *** MATH PACKAGES ***
%
%\usepackage{amsmath}
% A popular package from the American Mathematical Society that provides
% many useful and powerful commands for dealing with mathematics.
%
% Note that the amsmath package sets \interdisplaylinepenalty to 10000
% thus preventing page breaks from occurring within multiline equations. Use:
%\interdisplaylinepenalty=2500
% after loading amsmath to restore such page breaks as IEEEtran.cls normally
% does. amsmath.sty is already installed on most LaTeX systems. The latest
% version and documentation can be obtained at:
% http://www.ctan.org/pkg/amsmath





% *** SPECIALIZED LIST PACKAGES ***
%
%\usepackage{algorithmic}
% algorithmic.sty was written by Peter Williams and Rogerio Brito.
% This package provides an algorithmic environment fo describing algorithms.
% You can use the algorithmic environment in-text or within a figure
% environment to provide for a floating algorithm. Do NOT use the algorithm
% floating environment provided by algorithm.sty (by the same authors) or
% algorithm2e.sty (by Christophe Fiorio) as the IEEE does not use dedicated
% algorithm float types and packages that provide these will not provide
% correct IEEE style captions. The latest version and documentation of
% algorithmic.sty can be obtained at:
% http://www.ctan.org/pkg/algorithms
% Also of interest may be the (relatively newer and more customizable)
% algorithmicx.sty package by Szasz Janos:
% http://www.ctan.org/pkg/algorithmicx




% *** ALIGNMENT PACKAGES ***
%
%\usepackage{array}
% Frank Mittelbach's and David Carlisle's array.sty patches and improves
% the standard LaTeX2e array and tabular environments to provide better
% appearance and additional user controls. As the default LaTeX2e table
% generation code is lacking to the point of almost being broken with
% respect to the quality of the end results, all users are strongly
% advised to use an enhanced (at the very least that provided by array.sty)
% set of table tools. array.sty is already installed on most systems. The
% latest version and documentation can be obtained at:
% http://www.ctan.org/pkg/array


% IEEEtran contains the IEEEeqnarray family of commands that can be used to
% generate multiline equations as well as matrices, tables, etc., of high
% quality.




% *** SUBFIGURE PACKAGES ***
%\ifCLASSOPTIONcompsoc
%  \usepackage[caption=false,font=normalsize,labelfont=sf,textfont=sf]{subfig}
%\else
%  \usepackage[caption=false,font=footnotesize]{subfig}
%\fi
% subfig.sty, written by Steven Douglas Cochran, is the modern replacement
% for subfigure.sty, the latter of which is no longer maintained and is
% incompatible with some LaTeX packages including fixltx2e. However,
% subfig.sty requires and automatically loads Axel Sommerfeldt's caption.sty
% which will override IEEEtran.cls' handling of captions and this will result
% in non-IEEE style figure/table captions. To prevent this problem, be sure
% and invoke subfig.sty's "caption=false" package option (available since
% subfig.sty version 1.3, 2005/06/28) as this is will preserve IEEEtran.cls
% handling of captions.
% Note that the Computer Society format requires a larger sans serif font
% than the serif footnote size font used in traditional IEEE formatting
% and thus the need to invoke different subfig.sty package options depending
% on whether compsoc mode has been enabled.
%
% The latest version and documentation of subfig.sty can be obtained at:
% http://www.ctan.org/pkg/subfig




% *** FLOAT PACKAGES ***
%
%\usepackage{fixltx2e}
% fixltx2e, the successor to the earlier fix2col.sty, was written by
% Frank Mittelbach and David Carlisle. This package corrects a few problems
% in the LaTeX2e kernel, the most notable of which is that in current
% LaTeX2e releases, the ordering of single and double column floats is not
% guaranteed to be preserved. Thus, an unpatched LaTeX2e can allow a
% single column figure to be placed prior to an earlier double column
% figure.
% Be aware that LaTeX2e kernels dated 2015 and later have fixltx2e.sty's
% corrections already built into the system in which case a warning will
% be issued if an attempt is made to load fixltx2e.sty as it is no longer
% needed.
% The latest version and documentation can be found at:
% http://www.ctan.org/pkg/fixltx2e


%\usepackage{stfloats}
% stfloats.sty was written by Sigitas Tolusis. This package gives LaTeX2e
% the ability to do double column floats at the bottom of the page as well
% as the top. (e.g., "\begin{figure*}[!b]" is not normally possible in
% LaTeX2e). It also provides a command:
%\fnbelowfloat
% to enable the placement of footnotes below bottom floats (the standard
% LaTeX2e kernel puts them above bottom floats). This is an invasive package
% which rewrites many portions of the LaTeX2e float routines. It may not work
% with other packages that modify the LaTeX2e float routines. The latest
% version and documentation can be obtained at:
% http://www.ctan.org/pkg/stfloats
% Do not use the stfloats baselinefloat ability as the IEEE does not allow
% \baselineskip to stretch. Authors submitting work to the IEEE should note
% that the IEEE rarely uses double column equations and that authors should try
% to avoid such use. Do not be tempted to use the cuted.sty or midfloat.sty
% packages (also by Sigitas Tolusis) as the IEEE does not format its papers in
% such ways.
% Do not attempt to use stfloats with fixltx2e as they are incompatible.
% Instead, use Morten Hogholm'a dblfloatfix which combines the features
% of both fixltx2e and stfloats:
%
% \usepackage{dblfloatfix}
% The latest version can be found at:
% http://www.ctan.org/pkg/dblfloatfix




%\ifCLASSOPTIONcaptionsoff
%  \usepackage[nomarkers]{endfloat}
% \let\MYoriglatexcaption\caption
% \renewcommand{\caption}[2][\relax]{\MYoriglatexcaption[#2]{#2}}
%\fi
% endfloat.sty was written by James Darrell McCauley, Jeff Goldberg and 
% Axel Sommerfeldt. This package may be useful when used in conjunction with 
% IEEEtran.cls'  captionsoff option. Some IEEE journals/societies require that
% submissions have lists of figures/tables at the end of the paper and that
% figures/tables without any captions are placed on a page by themselves at
% the end of the document. If needed, the draftcls IEEEtran class option or
% \CLASSINPUTbaselinestretch interface can be used to increase the line
% spacing as well. Be sure and use the nomarkers option of endfloat to
% prevent endfloat from "marking" where the figures would have been placed
% in the text. The two hack lines of code above are a slight modification of
% that suggested by in the endfloat docs (section 8.4.1) to ensure that
% the full captions always appear in the list of figures/tables - even if
% the user used the short optional argument of \caption[]{}.
% IEEE papers do not typically make use of \caption[]'s optional argument,
% so this should not be an issue. A similar trick can be used to disable
% captions of packages such as subfig.sty that lack options to turn off
% the subcaptions:
% For subfig.sty:
% \let\MYorigsubfloat\subfloat
% \renewcommand{\subfloat}[2][\relax]{\MYorigsubfloat[]{#2}}
% However, the above trick will not work if both optional arguments of
% the \subfloat command are used. Furthermore, there needs to be a
% description of each subfigure *somewhere* and endfloat does not add
% subfigure captions to its list of figures. Thus, the best approach is to
% avoid the use of subfigure captions (many IEEE journals avoid them anyway)
% and instead reference/explain all the subfigures within the main caption.
% The latest version of endfloat.sty and its documentation can obtained at:
% http://www.ctan.org/pkg/endfloat
%
% The IEEEtran \ifCLASSOPTIONcaptionsoff conditional can also be used
% later in the document, say, to conditionally put the References on a 
% page by themselves.




% *** PDF, URL AND HYPERLINK PACKAGES ***
%
%\usepackage{url}
% url.sty was written by Donald Arseneau. It provides better support for
% handling and breaking URLs. url.sty is already installed on most LaTeX
% systems. The latest version and documentation can be obtained at:
% http://www.ctan.org/pkg/url
% Basically, \url{my_url_here}.




% *** Do not adjust lengths that control margins, column widths, etc. ***
% *** Do not use packages that alter fonts (such as pslatex).         ***
% There should be no need to do such things with IEEEtran.cls V1.6 and later.
% (Unless specifically asked to do so by the journal or conference you plan
% to submit to, of course. )


% correct bad hyphenation here
\hyphenation{op-tical net-works semi-conduc-tor}

\usepackage[slovene]{babel}
\usepackage[utf8]{inputenc}
\usepackage[T1]{fontenc}
\usepackage{lmodern}

% Te pakete sem jaz sam vključil:
\usepackage{verbatim} % za bločne komentarje
\usepackage{amsfonts} % za številske množice
\usepackage{amsmath} % za krepke simbole

\begin{document}
%
% paper title
% Titles are generally capitalized except for words such as a, an, and, as,
% at, but, by, for, in, nor, of, on, or, the, to and up, which are usually
% not capitalized unless they are the first or last word of the title.
% Linebreaks \\ can be used within to get better formatting as desired.
% Do not put math or special symbols in the title.
\title{Slikovno vodena radioterapija glave in vratu: pregled področja}
%
%
% author names and IEEE memberships
% note positions of commas and nonbreaking spaces ( ~ ) LaTeX will not break
% a structure at a ~ so this keeps an author's name from being broken across
% two lines.
% use \thanks{} to gain access to the first footnote area
% a separate \thanks must be used for each paragraph as LaTeX2e's \thanks
% was not built to handle multiple paragraphs
%

\author{Domen~Močnik,~\IEEEmembership{Fakulteta~za~elektrotehniko,~Univerza~v~Ljubljani,~Tržaška~25,~SI-1000,}~SLOVENIJA}% <-this % stops a space

% note the % following the last \IEEEmembership and also \thanks - 
% these prevent an unwanted space from occurring between the last author name
% and the end of the author line. i.e., if you had this:
% 
% \author{....lastname \thanks{...} \thanks{...} }
%                     ^------------^------------^----Do not want these spaces!
%
% a space would be appended to the last name and could cause every name on that
% line to be shifted left slightly. This is one of those "LaTeX things". For
% instance, "\textbf{A} \textbf{B}" will typeset as "A B" not "AB". To get
% "AB" then you have to do: "\textbf{A}\textbf{B}"
% \thanks is no different in this regard, so shield the last } of each \thanks
% that ends a line with a % and do not let a space in before the next \thanks.
% Spaces after \IEEEmembership other than the last one are OK (and needed) as
% you are supposed to have spaces between the names. For what it is worth,
% this is a minor point as most people would not even notice if the said evil
% space somehow managed to creep in.



% The paper headers
%\markboth{Journal of \LaTeX\ Class Files,~Vol.~14, No.~8, August~2015}%
%{Shell \MakeLowercase{\textit{et al.}}: Bare Demo of IEEEtran.cls for IEEE Journals}
% The only time the second header will appear is for the odd numbered pages
% after the title page when using the twoside option.
% 
% *** Note that you probably will NOT want to include the author's ***
% *** name in the headers of peer review papers.                   ***
% You can use \ifCLASSOPTIONpeerreview for conditional compilation here if
% you desire.




% If you want to put a publisher's ID mark on the page you can do it like
% this:
%\IEEEpubid{0000--0000/00\$00.00~\copyright~2015 IEEE}
% Remember, if you use this you must call \IEEEpubidadjcol in the second
% column for its text to clear the IEEEpubid mark.



% use for special paper notices
%\IEEEspecialpapernotice{(Invited Paper)}




% make the title area
\maketitle

% As a general rule, do not put math, special symbols or citations
% in the abstract or keywords.
\begin{abstract}
The abstract goes here.
\end{abstract}

% Note that keywords are not normally used for peerreview papers.
%\begin{IEEEkeywords}
%IEEE, IEEEtran, journal, \LaTeX, paper, template.
%\end{IEEEkeywords}






% For peer review papers, you can put extra information on the cover
% page as needed:
% \ifCLASSOPTIONpeerreview
% \begin{center} \bfseries EDICS Category: 3-BBND \end{center}
% \fi
%
% For peerreview papers, this IEEEtran command inserts a page break and
% creates the second title. It will be ignored for other modes.
\IEEEpeerreviewmaketitle



\section{Introduction}
% The very first letter is a 2 line initial drop letter followed
% by the rest of the first word in caps.
% 
% form to use if the first word consists of a single letter:
% \IEEEPARstart{A}{demo} file is ....
% 
% form to use if you need the single drop letter followed by
% normal text (unknown if ever used by the IEEE):
% \IEEEPARstart{A}{}demo file is ....
% 
% Some journals put the first two words in caps:
% \IEEEPARstart{T}{his demo} file is ....
% 
% Here we have the typical use of a "T" for an initial drop letter
% and "HIS" in caps to complete the first word.
\IEEEPARstart{T}{his} demo file is intended to serve as a ``starter file''
for IEEE journal papers produced under \LaTeX\ using
IEEEtran.cls version 1.8b and later.
% You must have at least 2 lines in the paragraph with the drop letter
% (should never be an issue)
I wish you the best of success.

\hfill mds
 
\hfill August 26, 2015

\subsection{Subsection Heading Here}
Subsection text here.

% needed in second column of first page if using \IEEEpubid
%\IEEEpubidadjcol

\section{Uvod}

\subsection{Delovanje radioterapije}

Visoko energijsko obsevanje celice poškoduje njen genski material, t.j.~deoksiribonukleinsko kislino (DNK). Če je poškodba DNK tolikšna, da je celica ni zmožna več popraviti, celica izgubi svoj potencial za nadaljnjo delitev oz.~razmnoževanje in sčasoma odmre. Obsevanje lahko bodisi reagira neposredno s celično DNK in jo poškoduje, kar imenujemo neposredni učinek obsevanja, bodisi z ionizacijo ali vzbujanjem vodnih komponent celic povzroči proste radikale, ki nato poškudujejo celični DNK, kar imenujemo posredni učinek obsevanja.

Cilj radioterapije je zadosti poškodovati DNK rakavih celic, da se te prenehajo deliti in sčasoma odmrejo. Obsevanje rakavih celic poškoduje tudi normalne celice, ki se nahajajo v neposredni bližini ali pa na poti do rakavih celic. Čeprav sevanje poškoduje tako rakave kot normalne celice, je cilj radioterapije dostava čim večje doze sevanja rakavim celicam in hkrati čim bolj zmanjšati sevalno izpostavljenost normalnih celic.

Sevanje v radioterapiji je ionizirajoče, kar pomeni, da ga sestavljajo visoko energijski delci (fotoni ali pa ioni), ki pri potovanju skozi tkivo prenesejo dovolj energije na elektrone v atomih ali molekulah, da se osvobodijo, atomi in molekule v obsevanem tkivu se tako ionizirajo in sprožijo kemijske reakcije, ki posredno ali pa neposredno poškodujejo DNK celic. \emph{Linearni prenos energije} (LPE) je naglost sproščanja energije delca in ta doseže svoj maksimum (imenovan \emph{Braggov vrh}) malo pred koncem dosega poti delca. Lokacijo tega vrha je mogoče prilagoditi tako, da se največ energije sevanja sprosti v tumorju in čim manj v normalnemu tkivu. Biološka učinkovitost (ubijanje celic) radioterapije je tem boljša, čim večji je LPE delcev sevanja.

\subsection{Dostavljanje sevanja in tipi sevanja}

Najbolj pogost način obsevanja rakavega mesta je \emph{obsevanje z zunanjimi žarki}, kjer se dostavlja visoko energijske delce s snopi žarkov, ki izvirajo zunaj telesa. Drugi, manj pogost način dostave je \emph{notranje obsevanje}, kjer se vsadi radioaktivni izvor neposredno v tumor.

Tipe sevanja opredeljujejo delci v žarkih. Fotonsko sevanje prenaša energijo z brezmasnimi fotoni (elektromagnetno valovanje). X-žarki in $\gamma$-žarki so razpršena ionizirajoča fotonska sevanja z majhnim LPE. X-žarke ustvarjajo naprave, ki vzbujajo elektrone, to so katodne cevi in linearni pospeševalniki, medtem ko $\gamma$-žarki izvirajo iz radioaktivnega razpada snovi (kobalt-60, radij, cezij). Poleg fotonskih žarkov se nabolj pogosto uporablja še elektronske žarke, ki so posebej učinkoviti pri zdravljenju kožnega raka in tumorjev, ki so blizu površja telesa, kajti elektroni ne prodrejo globoko v tkivo. V sodobni radioterapiji se fotonsko obsevanje postopoma nadomešča z obsevanjem s težjimi delci, kot so protoni, nevtroni ter težki ioni (helij, ogljik, dušik, argon, neon). Tovrstno sevanje ima višji LPE kot fotoni, zato ima višjo biološko efektivnost. So pa naprave za proizvajanje sevanja s tovrstnimi delci (sinhrotroni in ciklotroni) precej dražje.

Protonski žarki so novejša oblika obsevanja z delci in zaradi svojevrstnega Braggovega vrha ponuja boljšo porazdelitev doze, t.j.~dopušča maksimalen prenos energije na tumor in povzroča minimalno škodo normalnemu tkivu na poti žarka. Uporablja se predvsem za zdravljenje pediatričnih tumorjev in tumorjev blizu kritičnih struktur, kot je hrbtenica, ali pa lobanjski tumorji,
kajti v teh primerih je prizanašanje normalnemu tkivu ključnega pomena. Ker ciklotroni postajajo cenovno bolj dostopni, lahko v prihodnosti pričakujemo širšo uporabo protonske terapije.

\subsection{Sodobne radioterapevtske tehnike}

\begin{comment}
Celice so zmožne popraviti manjše poškodbe DNK. \emph{Radioterapija po delih} izkorišča razliko v radiobiološki lastnosti normalnih in rakavih celic: normalne celice se v primerjavi z rakavimi celicami razmnožujejo počasneje, zato imajo več časa, da popravijo poškodbe DNK. Tovrstna radioterapija se izvaja periodično z vmesnimi odmori, ko imajo normalne celice čas, da obnovijo poškodovane DNK. Dnevne doze sevanja pri tovrstni radioterapiji so majhne, od 1.5 do 3 Gy, s čimer se v veliki meri izogne povzročitvi nepopravljive škode na normalnih celicah, je pa zato potrebno obsevanja ponavljati skoraj vsak dan v obdobju nekaj tednov, da se zadane celično smrt rakavim celicam.

Alternativa radioterapiji po delih je \emph{stereotaktična radioterapija}, kjer se dostavlja visoke doze sevanja v zgolj nekaj terapijah. Zaradi visoke doze sevanja je velika verjetnost, da bo celotno tkivo v obsevanem obmčju poškodovano, zato je ključnega pomena, da se obsevano območje prilagodi tumorju kar se da natančno. To nam omogočata \emph{intenzitetno modulirana radioterapija} (IMRT).

katere cilj je natančno dostaviti zelo visoko dozo sevanja v dobro definirano območje tumorja v zgolj nekaj terapijah. Zaradi visoke doze sevanja je zelo verjetno poškodovano celotno tkivo v ciljnem območju. Ker želimo prizadeti čim manj zdravega tkiva v neposredni okolici tumorja, hkrati pa uničiti celotno ... Za dobro prilagoditev obsevalnega območja tumorju se uporabljata tehniki \emph{moduliranja intenzitete} ter \emph{slikovno vodena radioterapija}. Intenzitetno modulirana radioterapija uporablja naprave, imenovane \emph{kolimatorji}
\end{comment}

Učinkovita sodobna radioterapija temelji na natančnem dostavljanu visoke doze sevanja. Zaradi visoke doze sevanja je velika verjetnost, da je celotno tkivo v obsevanem območju poškodovano, zato je ključnega pomena, da se obsevano območje prilagodi tumorju kar se da natančno in se s tem prizadane čim manj zdravega tkiva. To nam omogočata v nadaljevanju opisani tehniki.

Za \emph{intenzitetno modulirano radioterapijo} je značilno, da uporablja \emph{večlistnate kolimatorje}, to so naprave, ki so zmožne prihajajoče delce sevanja iz izvora preusmeriti v take snope, da skupaj oblikujejo kompleksno 3D obliko polja, v katero sevanje prenese svojo energijo (v nadaljevanju: \emph{radiacijsko polje}). S tem je mogoče oblikovati radiacijska polja, ki se dobro prilegajo tumorju. V ta namen se pred terapijo zajame CT ali MR sliko pacienta, ki skupaj z dodanimi informacijami služi kot načrt za radioterapijo. Med dodane informacije sodijo orisan bruto volumen tumorja (področje na sliki, za katerega je razvidno, da pripada tumorju), orisan volumen klinične tarče (bruto volumen tumorja skupaj z okolico subkliničnega širjenja bolezni) ter orisane morebitne kritične organe, ki jih sevanje ne sme poškodovati. Ker med postopkom slikanja in radioterapije lahko pride do manjših odstopanj v legi pacienta ter do anatomskih premikov, ki jih povzročajo npr.~dihanje, požiranje, peristaltika, itd., zajema končni volumen načrtovane terapije poleg volumna klinične tarče še dodatno okolico, s katero je zagotovljeno, da je pokrit celoten tumor. Na podlagi dobljenega načrta radioterapevtske naprave ustvarijo radiacijsko polje.

Sodobne naprave za radioterapijo združujejo tudi CT skenerje, s katerimi je mogoče posneti CT sliko pacienta nekaj trenutkov pred ali pa kar med radioterapijo in se uporabi za izboljšanje natančnosti sevanja. CT ali MR sliko, ki je bila zajeta za načrtovanje (običajno že nekaj dni pred radioterapijo), se z avtomatskimi postopki poravna na CT sliko, zajeto tik pred ali pa med radioterapijo. S tem poravnamo tudi orisane volumne in tako oblikovanje sevanja sproti prilagajamo trenutni legi pacienta in njegovih anatomskih struktur. Tovrstno tehniko imenujemo \emph{slikovno vodena radioterapija}. Premiki anatomskih struktur zaradi dihanja, požiranja, peristaltike, mišičnega upogibanja in prostovoljnih premikov, ki so lahko posledica bolečin ali nelagodja, med radioterapijo segajo od nekaj milimetrov pa do nekaj centimetrov. Skrajna primera sta gibanje pljučnega tumorja med dihanjem (do 3cm, čeprav bolj tipične vrednosti znašajo okoli 1cm) in pa premiki tumorja na prostati (do 1cm) zaradi spremembe rektalne vsebine. Če se radioterapija ponavlja v več fazah tekom več dni, je potrebno upoštevati tudi skrčenje tumorja, kar je posledica prejšnjih radioterapij, in pa morebitno izgubo telesne mase pacienta zaradi bolezni. Ker torej poravnava slik lahko bistveno pripomore k natančnosti slikovno vodene radioterapije, je smiselno izboljševati natančnost poravnave slik. Pri merjenju naključnih premikov tkiv znotraj telesa je bilo ugotovljeno, da izboljšava v natančnosti določitve radiacijskega polja zgolj za 1--2 mm ne prinese bistvenega dosežka. 

\section{Postopki za avtomatsko poravnavo medicinskih slik v radioterapiji glave in vratu}

Slikovna poravnava ali \emph{registracija} je iskanje prostorskih preslikav med dvema ali več zajetimi slikami, ki med njimi vzpostavi smiselno anatomsko ali funkcionalno korespondenco. Registracijo se poleg že prej opisane uporabe v radioterapiji na podoben način uporablja še pri nekaterih operacijskih posegih za medoperacijsko posodabljanje in izboljševanje predoperacijskih načrtov. Namen registracije je lahko tudi omogočiti zdravniku boljše analitske in diagnostične zmožnosti, če poravnamo slike iste osebe, zajete z različnimi protokoli ali pa ob različnih trenutkih. V prvem primeru se uporabi različne slikovne modalitete, s čimer se pridobi komplementarne anatomske ali funkcionalne informacije in se jih v postopku registracije združi, v drugem primeru pa se opazuje spremembe oblike tumorja ali pa različne anatomske spremembe skozi časovno obdobje.

V tem poročilu se bomo omejili na poravnavo dveh slik. Ena izmed njiju se običajno imenuje \emph{izvor} ali \emph{gibljiva slika}, druga pa \emph{tarča} ali \emph{negibna slika}. Gibljiva slika je podana s preslikavo $S\colon\Omega_S\subset\mathbb{R}^d\to\mathbb{R}$, negibna slika pa s $T\colon\Omega_T\subset\mathbb{R}^d\to\mathbb{R}$, $d\in\{2,3\}$. $\Omega_S$ in $\Omega_T$ sta slikovni domeni, medtem ko vrednosti preslikav predstavljajo slikovne intenzitete. Cilj registracije je poiskati preslikavo $W\colon\Omega_S\to\Omega_T$, ki kar se da dobro poravna sliki. Preslikavo $W$ običajno zapišemo v obliki
\begin{equation}
 W(\boldsymbol{x}) = \boldsymbol{x} + \boldsymbol{u}(\boldsymbol{x}),
\end{equation}
kjer je $\boldsymbol{u}$ vektorsko polje pomikov.

Algoritem za registracijo sestoji iz treh ključnih komponent: \emph{deformacijski model}, \emph{kriterijska funkcija} in pa \emph{optimizacijska metoda}. V nadaljevanju se bomo posvetili vsaki izmed njih posebaj.

\subsection{Deformacijski model}

Deformacijski model je predpis, s katerim podamo množico vseh možnih preslikav, med katerimi algoritem potem skuša poiskati optimalno. Če se da to množico parametrizirati s kako podmnožico $\Theta\subset\mathbb{R}^n$, potem rečemo, da je model \emph{$n$-parametričen} ali pa da ima $n$ \emph{prostostnih stopenj}. Če iščemo optimalno preslikavo recimo v množici vseh afinih preslikav iz $\mathbb{R}^3$ v $\mathbb{R}^3$, potem je ta model 9-parametričen, saj se ga da opisati z devetimi realnimi parametri: trije določajo kote rotacij okoli vsake od koordinatnih osi, trije določajo skaliranje, trije pa komponente translacije. Večje, kot je število prostostnih stopenj, večji spekter možnih preslikav zajame model, vendar pa iskanje optimalne preslikave v modelu tedaj postane bolj zapleteno in računsko bolj zahtevno. Izbira kompleksnosti modela je odvisna od aplikacije; za registracijo slik mehkih tkiv so potrebni visoko-dimenzionalni modeli, ki lahko zajemejo večje lokalne variabilnosti, med tem ko za toge strukture, kot so kosti, zadoščajo nizko-dimenzionalni modeli.

Deformacijske modele v grobem razvrščamo v dve skupini: Modeli, izpeljani iz fizikalnih modelov, ter modeli iz teorije interpolacije. Pri fizikalnih modelih dobimo vektorsko polje pomikov kot rešitev parcialnih diferencialnih enačb, ki izhajajo iz mehanike elastičnih teles, mehanike fluidov, difuzijskih modelov in ostalo. Geometrijske preslikave, ki pripadajo modelom iz teorije interpolacije, tvorijo vektorsko utežene linearne kombinacije bodisi \emph{radialnih baznih funkcij}, \emph{zlepkov} (npr. kubični B-zlepki) oz.~njihovih tenzorskih produktov ali pa baznih funkcij iz Fourierove in valčne analize. Parametri v tovrstnih modelih so vektorske uteži oz.~premiki posameznih vozlišč iz slikovne domene. Pomiki vseh ostalih točk, ki niso vozlišča, so podani z interpolacijo. \emph{Lokalno afini modeli} poiščejo geometrijske preslikave tako, da domeno slike razdelijo na posamezne manjše kose in najprej za njih poišče optimalne afine preslikave, nato pa jih z interpolacijskimi metodami povežejo v skupno globalno preslikavo.

Deformacijski modeli lahko dodatno izkoriščajo predhodno znane informacije, recimo statistične informacije o deformacijskih poljih za določeno populacijo. Na podlagi teh informacij se tvori \emph{statistične deformacijske modele}, katerih prednost je, da imajo manj prostostnih stopenj kot splošni modeli. Vendar pa se tovrstne metode zanašajo na to, da so bile potrebne informacije pridobljene iz statistično reprezentativnega vzorca.

Pogosto so za deformacijske modele zaželjene dodatne lastnosti. Za odražanje realističnih deformacij je recimo smiselno zahtevati, da so preslikave iz deformacijskega modela homeomorfizmi (zvezne bijektivne preslikave z zveznim inverzom). To lastnost v literaturi pogosto imenujejo \emph{ohranjanje topologije}. Še močnejša zahteva je, da so preslikave difeomorfizmi (zvezno odvedljivi homeomorfizmi z zvezno odvedljivimi inverzi), s čimer dobimo določeno stopnjo gladkosti preslikav. Še ena lastnost, ki je sicer bolj lastnost celotnega algoritma in ne zgolj deformacijskega modela, je \emph{simetričnost}. Če pri algoritmu za registracijo zamenjamo sliki, tako da izvor postane tarča in obratno, in pri tem dobimo inverzno preslikavo, potem je algoritem \emph{simetričen}, sicer je \emph{asimetričen}.

\subsection{Kriterijska funkcija}

Kriterijska funkcija je funkcija, ki kvantitativno ovrednoti, kako dobro sta sliki poravnani. Če je model parametriziran z množico $\Theta$, potem se pričakuje, da kriterijska funkcija $M\colon\Theta\to\mathbb{R}$ zavzame svoj minimum pri tistem parametru $\boldsymbol{\theta}\in\Theta$, ki v danem modelu ponuja najboljšo poravnavo slik. Ker definiranje pojma \textit{najboljša poravnava} predstavlja precejšen izziv, je tudi izbira ustrezne kriterijske funkcije v praksi precej težek problem. Običajno se kriterijski funkciji prišteje še t.~i.~\emph{regularizacijski člen} $R$, ki odvrača nerealistične preslikave, in se nato poišče
\begin{equation}
 \arg\min_{\boldsymbol{\theta}}M\big(T,S\circ W(\boldsymbol{\theta})\big)+R\big(W(\boldsymbol{\theta})\big).
\end{equation}


Kriterijske funkcije se v grobem delijo v dve skupini: \emph{geometrijske} in \emph{ikonične}. V prvem primeru se najprej z avtomatskimi ali pa ročnimi postopki na obeh slikah poiščejo izstopajoče oslonilne točke, kriterijska funkcija pa potem med njimi skuša poiskati korespondenčne pare na podlagi opisnikov oslonilnih točk. Tovrstne kriterijske funkcije pridejo v upoštev pri modelih z radialnimi baznimi funkcijami, ki tvorijo transformacijo s interpolacijo na osnovi oslonilnih točk. Namesto oslonilnih točk se lahko, če so na voljo, uporabi tudi kar njihove verjetnostne porazdelitve, ali pa značilne regije, ploskve in linije. Ikonične kriterijske funkcije za kriterij podobnosti izkoriščajo informacijo iz intenzitet vseh slikovnih elementov. Izbira ikoničnih kriterijskih funkcij mora upoštevati, ali imamo pri poravnavi opravka z monomodalnim primerom (obe sliki sta zajeti po enakem protokolu, pridobljen je isti tip informacij) ali z multimodalnim primerom (sliki sta zajeti po različnem protokolu).

Pri poravnavi monomodalnih slik je kriterijska funkcija v najpreprostejšem primeru lahko zgolj vsota kvadratov razlik v intenziteti istoležečih vokslov obeh slik. Izboljšane funkcije uporabljajo večdimenzionalne statistike, imenovane \emph{značilnice}, prirejene vsakemu slikovnemu elementu posebaj, in vsebujejo informacije o izgledu okolice posameznega slikovnega elementa. Poravnava multimodalnih slik zahteva bolj dovršene kriterijske funkcije. Te običajno izvirajo iz teorije informacij in uporabljajo korelacijsko razmerje, medsebojno informacijo ali Kullback-Leiblerjevo divergenco. Manj pogost pristop je prevedba slik v isto modaliteto in nato uporaba kriterijskih funkcij za monomodalne poravnave.

\subsection{Optimizacijske metode}

Ko sta deformacijski model in kriterijska funkcija izbrana, preostane računsko najbolj zahteven del registracije, to je iskanje parametrov, ki optimizirajo kriterijsko funkcijo. Kriterijske funkcije so običajno nelinearne in nekonveksne, kar nas omeji na iskanje zgolj lokalnih ekstremov, optimizacija pa v teh primerih postane slabo pogojen problem, kar pomeni, da že majhne variacije v vhodnih podatkih (slikah) ali začetnih približkih lahko zelo vplivajo na končni rezultat (poravnava). Dodatno je postopek optimizacije otežen, če ima model visoko število prostostnih stopenj.

Če optimizacijska metoda išče optimalni parameter zgolj v diskretni podmnožici množice $\Theta$, potem  jo uvrščamo med diskretne metode, sicer pa med zvezne metode.

Skupno zveznim optimizacijskim metodam je, da iščejo optimalni parameter iterativno
\begin{equation}
 \boldsymbol{\theta}_{k+1}=\boldsymbol{\theta}_k+\alpha_k\boldsymbol{g}_k(f(\boldsymbol{\theta}_k)),
\end{equation}
kjer je $k\in\mathbb{N}$ korak iteracije, $\boldsymbol{\theta}_k$ $k$-ti približek optimalnega parametra, $\alpha_k$ realno število, $f=M+R$ kriterijska funkcija ter $\boldsymbol{g}_k$ smer iskanja. \emph{Gradientne metode} za $\boldsymbol{g}$ vzamejo smer največjega spusta, $\boldsymbol{g}_k = -\nabla_{\boldsymbol{\theta}}f(\boldsymbol{\theta_k})$. \emph{Metode konjugiranih gradientov} pospešijo konvergenco z izbiro $\boldsymbol{g}_k = -\nabla_{\boldsymbol{\theta}}f(\boldsymbol{\theta_k})+\beta_k\boldsymbol{g}_{k-1}$, kjer je potrebno še primerno izbrati realne parametre $\beta_k$. \emph{Kvazi-Newtonove metode} iščejo stacionarne točke kot ničle gradienta s posplošeno Newtonovo metodo, tako da je $\boldsymbol{g}=-\hat{H}^{-1}(\boldsymbol{\theta})\nabla_{\boldsymbol{\theta}}f(\boldsymbol{\theta})$, kjer je matrika $\hat{H}$ približek za Hessejevo matriko drugih odvodov funkcije $f$. Računanje odvodov je računsko zahteven postopek, če dimenzionalnost prostora parametričnega prostora $\Theta$ velika, zato se v tovrstnih primerih uporablja \emph{stohastične gradientne metode}, kjer se namesto gradienta uporabi približek za gradient, izračunan z računsko manj zahtevnimi postopki.

Nekatere diskretne optimizacijske metode delujejo tako, da prevedejo optimizacijski problem na problem maksimalnega pretoka oz.~minimalnega prereza v primerno skonstruiranem uteženem grafu. Tovrsten problem je dobro poznan problem iz linearne optimizacije in se ga da učinkovito rešiti z algoritmom Edmonds-Karp. Preostale diskretne metode so \emph{metode širjenje verjetja} in pa metode, ki dani problem skušajo simulirati s preprostejšim problemom, rešljivim z linearnim programiranjem. Pomankljivost diskretnih metode je, da išče optimalne parametre le v naprej definiranih diskretnih podmnožicah parametrov, torej so manj natančne, so pa zato računsko bolj učinkovite.


\subsection{Vrednotenje kakovosti poravnave slik}

Pri kvantitativnem vrednotenju kakovosti avtomatske poravnave slik veliko težavo predstavlja nepoznavanje resnične preslikave, zaradi česar je, razen v sintetičnih primerih, nemogoče primerjati izračunano in resnično preslikavo. Zato se metode za vrednotenje v praksi zanašajo na referenčne podatke, ki jih ročno vnesejo strokovnjaki.

Najbolj pogosti metodi za kvantitativno vrednotenje kakovosti poravnave slik sta: Strokovnjak na slikah (na izvoru in na tarči)
\begin{enumerate}
\item označi $n$ parov oslonilnih točk $(\lambda_i,\kappa_i)\in\Omega_S\times\Omega_T$, ($i=1,\dots,n$), tako da točki iz para $(\lambda_i, \kappa_i)$ označujeta isto anatomsko lokacijo, vidno na obeh slikah. Kakovost poravnave se potem meri s statistikami, ki vključujejo evklidske razdalje $d_i=\|W(\boldsymbol{\lambda}_i)-\boldsymbol{\kappa}_i\|_2$, recimo njihovo povprečje in standardna deviacija, minimum, maksimum idr.
\item oriše območji zanimanja $A\subset\Omega_S$, $B\subset\Omega_T$, recimo katerega od organov ali pa tumor. Kakovost poravnave območja zanimanja $A$ se meri z merami, kot so Dice, Hausdorffova razdalja idr., ki vrednotijo prekrivanje originalnega področja $B$ in preslikanega območja $W(A)$.
\end{enumerate}

\subsection{Monomodalne poravnave}

V delu \cite{hardcastle2012} so avtorji ocenjevali uspešnost slikovne poravnave, kjer so bile tako slike za načrtovanje kot tudi slike za določitev lege zajete s spiralnim CT skenerjem. Avtorji so imeli na voljo 22 parov slik različnih pacientov z neoplazmami na področju glave in vratu. Na slikah je izkušeni zdravnik orisal bruto volumen tumorja ter naslednje kritične organe: hrbtenjačo, možgansko deblo in parotidni žlezi. Za slikovno poravnavo so uporabili večmrežno metodo Demons\cite{vercauteren2009}, ki išče deformacijsko polje kot rešitev difuzijske enačbe, in pa metodo za poravnavo na podlagi perečih značilnic (angl.~salient-feature-based registration, SFBR), ki uporablja interpolacijo s TPS zlepki (angl.~thin plate splines). Kakovost poravnave orisanih območij je bila kvantitativno ovrednotena z mero Dice in srednjo vrednost Housdorffovih razdalij po rezinah (angl.~mean Hausdorff distance, MHSD), strokovnjak pa je poravnavo ocenil še kvalitativno. Edina statistično značilna razlika uspešnosti med obema algoritmoma je bila opažena za možgansko deblo, kjer je SFBR dosegel višji Dice ($p=0.001$) in nižji MSHD ($p=0.002$). Za $216$ preslikanih območij je strokovnjak ocenil, da se jih je $78$ preslikalo dobro in ne potrebujejo popravkov, 124 jih je uporabnih, a bi potrebovali manjše popravke, $14$ pa bi jih potrebovalo večje popravke in so nesprejemljivi. V 12 od teh 14 primerov je šlo za GTV. Avtorji so zaključili, da je uporaba omenjenih algoritmov za preslikavo klinično sprejemljiva, vseeno pa predlagajo, da rezultate pazljivo pregledajo strokovnjaki.

V \cite{hou2011} so raziskovali uspešnost večmrežnega algoritma Demons na slikah dvanajstih pacientov z rakom na področju glave in vratu. Za načrtovanje radioterapije so bile uporabljene običajne CT slike, v 5--7 tedenskem obdobju radioterapije pa je bilo za vsakega pacienta zajetih po več CBCT slik. Ker kriterijska funkcija metode primerja intenzitete slik in ker imajo lahko CBCT in CT slike različne intenzitete, je bilo potrebno le te pred poravnavo normalizirati. Za kvantitativno vrednotenje poravnave so na vseh slikah označili 9 oslonilnih točk in dve nenavedeni območji, nato pa so analizirali evklidske razdalje poravnanih in originalnih oslonilnih točk ter Dice mero poravnanih in originalnih območij. Povprečna razdalja oslonilnih točk je znašala $2{,}8\,{\pm}\,0{,}2$ mm za točke na mehkih tkivih in $2{,}4\,{\pm}\,0{,}2$ za točke na kostnih strukturah, kar je manj kot pa je največja razsežnost slikovnega elementa ($3$ mm). Povprečna vrednost Dice za orisane strukture je znašala $0,762\,\pm\,0,046$.

Tudi v raziskavi \cite{mencarelli2014} so izvajali poravnavo CT slik na CBCT, vendar so izbrali algoritem, ki za svoj model uporablja B-zlepke, za kriterijsko funkcijo uporablja korelacijski koeficient, optimizijska metoda pa je multiresolucijska gradientna metoda. Študija je zajemala 13 pacientov, ki so se zdravili z radioterapijo v obdobju 6 do 7 tednov, v katerem je bilo povprečno zajetih 35 CBCT slik na pacienta. Pacientom so pred slikanjem v rob tumorja transoralno vsadili od 4 do 10 zlatih označevalcev velikosti $0{,}35\times 2$, ki so kasneje na slikah služili kot oslonilne točke, ročno pa so na vsaki sliki označili še 15 oslonilnih točk na normalnem tkivu. Na CT slikah so pred poravnavo digitalno odstranili označevalce, da ti niso vplivali na poravnavo. Povprečna razdalja preslikanih lokacij označevalcev od originalnih je bila $3{,}3\,\pm\,0{,}38$, razdalja preslikanih oslonilnih točk od originalnih pa je bila $2{,}2\,\pm\,0{,}59$. Hkrati je bilo ugotovljeno, da se natančnost poravnave oslonilnih točk normalnega tkiva skozi obdobje radioterapije bistveno ne poslabša, medtem ko natančnost poravnave označevalcev značilno ($p<0{,}009$) pada za $0{,}21$ mm na teden, kar gre pripisati spremembi v velikosti in obliki tumorja kot odziv na radioterapijo (devetim pacientom se je tumor krčil, trem se je povečeval, pri enem pa ni bilo opazne razlike). Zato so avtorji zaključili, da je potrebna previdnost, ko se uporablja poravnavo slik za namen izračunavanja doze sevanja za tumor.





\begin{comment}
Za kvantitativno ovrednotenje kvalitete poravnave se uporablja več različnih kriterijev. Med najpogosteje uporabljenimi sta mera DICE (angl. DICE similarity measure) ter modificirana Hausdorffova razdalja, ki računa razdaljo med ploskvami segmentiranih anatomskih struktur:
\begin{equation}
 H = frac{1}{n}\min_{x\in A,\ y\in B} d(x,y)
\end{equation}
Mera DICE je izražena z enačbo
\begin{equation}
 D = \frac{2|A\cap B|}{|A| + |B|},
\end{equation}
kjer $|\cdot|$ označuje moč množice, $A$ je množica vokslov, ki pipada prvi segmentirani anatomski strukturi, $B$ pa je množica vokslov, ki pipada drugi anatomski strukturi. DICE doseže vrednost 1,
ko se anatomski strukturi na sliki popolnoma prekrivata, nižje vrednosti pa pomenijo slabšo poravnavo. Hausdorffova razdalja pa doseže nizke vrednosti za popolne poravnave ter višje vrednosti za nepopolne; idealno je 0. Rezultati v meri dice se lahko precej razlikujejo glede na različne anatomske strukture. Za strukture, ki imajo večje razmerje med površino in volumnom (npr. hrbtenjača) je značilno, da je mera dice nekoliko nižja, zato tu ne gre pričakovati visokih rezultatov. Za tovrstne anatomske strukture je bolje uporabiti Hausdorffovo razdaljo.
\end{comment}



% An example of a floating figure using the graphicx package.
% Note that \label must occur AFTER (or within) \caption.
% For figures, \caption should occur after the \includegraphics.
% Note that IEEEtran v1.7 and later has special internal code that
% is designed to preserve the operation of \label within \caption
% even when the captionsoff option is in effect. However, because
% of issues like this, it may be the safest practice to put all your
% \label just after \caption rather than within \caption{}.
%
% Reminder: the "draftcls" or "draftclsnofoot", not "draft", class
% option should be used if it is desired that the figures are to be
% displayed while in draft mode.
%
%\begin{figure}[!t]
%\centering
%\includegraphics[width=2.5in]{myfigure}
% where an .eps filename suffix will be assumed under latex, 
% and a .pdf suffix will be assumed for pdflatex; or what has been declared
% via \DeclareGraphicsExtensions.
%\caption{Simulation results for the network.}
%\label{fig_sim}
%\end{figure}

% Note that the IEEE typically puts floats only at the top, even when this
% results in a large percentage of a column being occupied by floats.


% An example of a double column floating figure using two subfigures.
% (The subfig.sty package must be loaded for this to work.)
% The subfigure \label commands are set within each subfloat command,
% and the \label for the overall figure must come after \caption.
% \hfil is used as a separator to get equal spacing.
% Watch out that the combined width of all the subfigures on a 
% line do not exceed the text width or a line break will occur.
%
%\begin{figure*}[!t]
%\centering
%\subfloat[Case I]{\includegraphics[width=2.5in]{box}%
%\label{fig_first_case}}
%\hfil
%\subfloat[Case II]{\includegraphics[width=2.5in]{box}%
%\label{fig_second_case}}
%\caption{Simulation results for the network.}
%\label{fig_sim}
%\end{figure*}
%
% Note that often IEEE papers with subfigures do not employ subfigure
% captions (using the optional argument to \subfloat[]), but instead will
% reference/describe all of them (a), (b), etc., within the main caption.
% Be aware that for subfig.sty to generate the (a), (b), etc., subfigure
% labels, the optional argument to \subfloat must be present. If a
% subcaption is not desired, just leave its contents blank,
% e.g., \subfloat[].


% An example of a floating table. Note that, for IEEE style tables, the
% \caption command should come BEFORE the table and, given that table
% captions serve much like titles, are usually capitalized except for words
% such as a, an, and, as, at, but, by, for, in, nor, of, on, or, the, to
% and up, which are usually not capitalized unless they are the first or
% last word of the caption. Table text will default to \footnotesize as
% the IEEE normally uses this smaller font for tables.
% The \label must come after \caption as always.
%
%\begin{table}[!t]
%% increase table row spacing, adjust to taste
%\renewcommand{\arraystretch}{1.3}
% if using array.sty, it might be a good idea to tweak the value of
% \extrarowheight as needed to properly center the text within the cells
%\caption{An Example of a Table}
%\label{table_example}
%\centering
%% Some packages, such as MDW tools, offer better commands for making tables
%% than the plain LaTeX2e tabular which is used here.
%\begin{tabular}{|c||c|}
%\hline
%One & Two\\
%\hline
%Three & Four\\
%\hline
%\end{tabular}
%\end{table}


% Note that the IEEE does not put floats in the very first column
% - or typically anywhere on the first page for that matter. Also,
% in-text middle ("here") positioning is typically not used, but it
% is allowed and encouraged for Computer Society conferences (but
% not Computer Society journals). Most IEEE journals/conferences use
% top floats exclusively. 
% Note that, LaTeX2e, unlike IEEE journals/conferences, places
% footnotes above bottom floats. This can be corrected via the
% \fnbelowfloat command of the stfloats package.







% Can use something like this to put references on a page
% by themselves when using endfloat and the captionsoff option.
\ifCLASSOPTIONcaptionsoff
  \newpage
\fi



% trigger a \newpage just before the given reference
% number - used to balance the columns on the last page
% adjust value as needed - may need to be readjusted if
% the document is modified later
%\IEEEtriggeratref{8}
% The "triggered" command can be changed if desired:
%\IEEEtriggercmd{\enlargethispage{-5in}}

% references section

% can use a bibliography generated by BibTeX as a .bbl file
% BibTeX documentation can be easily obtained at:
% http://mirror.ctan.org/biblio/bibtex/contrib/doc/
% The IEEEtran BibTeX style support page is at:
% http://www.michaelshell.org/tex/ieeetran/bibtex/
\bibliographystyle{IEEEtran}
% argument is your BibTeX string definitions and bibliography database(s)
%\bibliography{IEEEabrv,../bib/paper}
%
% <OR> manually copy in the resultant .bbl file
% set second argument of \begin to the number of references
% (used to reserve space for the reference number labels box)

\bibliography{literatura}





% that's all folks
\end{document}


